\section{Introduction}

Network security is a growing issue in the field of information assurance and cybersecurity. Network breaches can result in lost bussiness and revenue, loss of trust with clients, and the theft of personal data. The importance of secure networks necessitates new ways of thinking about how to secure and monitor network traffic from growing numbers of attacks. The types of network attacks are also growing, and the methods used to detect them need to be improved upon \cite{yang20}. Intrusion Detection Systems are a primary tool used to detect malicious network traffic, primarily by detecting signatures of attack traffic \cite{nguyen18}. Essentially, IDS are able to mine large sets of network traffic data in order to train a model to detect novel network data \cite{yang20}. Current IDS can fail to quickly and accurately identify attacks. There are many areas of study in how to improve current intrusion detection systems. In this paper, we will look at several studies that have applied concepts in artificial intelligence to this problem. 

In the first section of this paper, we discuss how the machine learning concepts of convolutional neural networks and long term short memory models can be effectively utilized to create more efficient, accurate, and robust IDS. We detail how this research has distinguished itself from previous research and demonstrated promising findings. (Talk about section 3). 

%\subsection{Research Hypotheses}
%\label{sec:hyp}
%
%\begin{enumerate}
%    \item Newly created coins are more likely to borrow code and or particular features from newer coins rather than older ones due to improvement in the cryptocurrenhttps://www.overleaf.com/project/5e18b94d620d5d00011732bdcy landscape as a whole.
%    \begin{enumerate}
%        \item Corollary: The developers of newer coins have an easier time implementing new features or design choices as compared to older coins.
%    \end{enumerate}
%    \item Newly created coins borrow code from Bitcoin over time as Bitcoin can provide a solid foundation for security properties and consensus mechanisms which are difficult to implement from scratch.
%    \item Older, better established coins such as Bitcoin borrow code or features from other newer coins.
%    \begin{enumerate}
%        \item Corollary: Newer coins are more likely to have either borrowed or developed new features which may be appealing to older coins.
%        \item Corollary: Key functions are copied often; less important functions are not copied at all. For example improvements to security models, enhanced cryptography, consensus models, 
%    \end{enumerate}
%\end{enumerate}

