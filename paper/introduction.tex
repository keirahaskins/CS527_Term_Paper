\section{Introduction}

%\subsection{Research Hypotheses}
%\label{sec:hyp}
%
%\begin{enumerate}
%    \item Newly created coins are more likely to borrow code and or particular features from newer coins rather than older ones due to improvement in the cryptocurrenhttps://www.overleaf.com/project/5e18b94d620d5d00011732bdcy landscape as a whole.
%    \begin{enumerate}
%        \item Corollary: The developers of newer coins have an easier time implementing new features or design choices as compared to older coins.
%    \end{enumerate}
%    \item Newly created coins borrow code from Bitcoin over time as Bitcoin can provide a solid foundation for security properties and consensus mechanisms which are difficult to implement from scratch.
%    \item Older, better established coins such as Bitcoin borrow code or features from other newer coins.
%    \begin{enumerate}
%        \item Corollary: Newer coins are more likely to have either borrowed or developed new features which may be appealing to older coins.
%        \item Corollary: Key functions are copied often; less important functions are not copied at all. For example improvements to security models, enhanced cryptography, consensus models, 
%    \end{enumerate}
%\end{enumerate}

