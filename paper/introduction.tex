\section{Introduction}

Open-source software is important in the creation and evolution of cryptocurrencies for a number of reasons; first, it allows for the sharing of innovative code design as well as the introduction of new security features into old as well as new cryptocurrencies. Second, it allows for the construction of trust and safety into the consensus mechanisms of cryptocurrencies as a whole. As the first cryptocurrency to be introduced in 2009, it makes sense that new coins would borrow various core components and features from Bitcoin. On the other hand it also makes sense that Bitcoin would potentially be interested in borrowing innovative features and design choices from newer coins.

\subsection{Research Hypotheses}
\label{sec:hyp}

\begin{enumerate}
    \item Newly created coins are more likely to borrow code and or particular features from newer coins rather than older ones due to improvement in the cryptocurrenhttps://www.overleaf.com/project/5e18b94d620d5d00011732bdcy landscape as a whole.
    \begin{enumerate}
        \item Corollary: The developers of newer coins have an easier time implementing new features or design choices as compared to older coins.
    \end{enumerate}
    \item Newly created coins borrow code from Bitcoin over time as Bitcoin can provide a solid foundation for security properties and consensus mechanisms which are difficult to implement from scratch.
    \item Older, better established coins such as Bitcoin borrow code or features from other newer coins.
    \begin{enumerate}
        \item Corollary: Newer coins are more likely to have either borrowed or developed new features which may be appealing to older coins.
        \item Corollary: Key functions are copied often; less important functions are not copied at all. For example improvements to security models, enhanced cryptography, consensus models, 
    \end{enumerate}
\end{enumerate}
%We make three main hypotheses in this paper; first, we hypothesize that new coins borrow code from newer coins. Second, we hypothesize that coins borrow code from Bitcoin. Lastly, we hypothesize that larger coins such as Bitcoin borrow code or features from other coins.
% Flush them out in a way that can be refuted. Someone should be able to understand what I'm getting at. "Oh that sounds interesting"

%they need to be more specific. work on saying exactly what you want to test.
% Why we are doing this, how do the hypotheses tie this together.

% don't forget to add verbiage about the versioning
% doing forget to add verbiage about copying specific bits (key functionality or extra features or something else)

(1) will show us whether or not newly created coins are in fact more likely to borrow code or features from newer coins rather than older ones. We hypothesize this because newer coins have the potential to make design decisions based upon the shortcomings of older coins in order to instead implement new features upon release of the currency.  

(2) will demonstrate whether or not it is the case that coins borrow base code from coins such as Bitcoin which are already well established. We hypothesize this because cryptocurrencies such as Bitcoin are already well established; they have already figured out how to design overall secure distributed systems.

(3) will demonstrate whether it is the case that coins such as Bitcoin borrow code or features from newer coins in an effort to improve their own code bases.