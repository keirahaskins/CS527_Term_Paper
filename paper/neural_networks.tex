\section{Neural Networks and Intrusion Detection Systems}
\citeN{choudhary09,su19,kumar11}. 

In this section, we discuss three papers that applied neural network models in order to create more accurate and efficient intrusion detection systems.  Su and Nguyen both created models based on convolutional neural networks, while Yang created a model utilizing Attention-Long Term Short Memory. 

Convolutional neural networks are used to analyze an image and extract features from that image, and is a common method in image processing (Nguyen). Previous research has utilized CNN to detect denial of service attacks on networks with lower computation times than previous machine learning methods (Nguyen). Previous research has also shown that a CNN is capable of transforming continuous network traffic into images that can then be accurately interpreted (Sun). Both Su and Nguyen utlize convolutional neural networks in this fashion. 

(Yang) also details the application of Long Term Short Memory (LTSM) models from previous research. LTSM is a model that was adapted from recurrent neural networks (RNN) that have been shown to perform better than traditional RNN, with the added ability to selectively forget certain information from a previous node based on some criteria (Yang).

All three of the papers detailed in this section use data sets provided by the Accossiation of Computing Machinery’s Special Interest Group on Knowledge Discovery and Data Mining (SIGKDD). This KDD data provides data network traffic data that the authors used to train and test their models (Yang, Su, Nguyen).  The researchers do not use the exact same data sets, but each use data sets from the same source containing similar data. 


\subsection{Convolutional Neural Networks}

In (Sun), the authors identified two problems with the current methods of monitoring for malicious network traffic: feature dependence and low accuracy. IDS models must be able to accurately identify and distinguish between the features of normal versus abnormal network traffic. There may also be multiple types of malicious traffic (including Denial of Service, Root to User, User to Root, and Probing attacks). The authors build on past research that has different methods of classification (such as k means clustering, principle component analysis, and support vector machines). These methods led to high false alarm rates. To better solve this problem of quickly and accurately identifying malicious traffic, the authors of this paper proposed a new deep learning model (named OCL) utilizing a convolutional neural network to extract features and long short-term memory (LTSM) to combine extracted features, which can then be classified. The authors show results that indicate their model achieves high accuracy, as well as better performance than other methods used in the past. 

To achieve better accuracy and increase efficiency, the authors proposed a model that can quickly and automatically identify important features of network traffic data. Their OCL model is divided in to three layers: data preprocessing, CNN, and LTSM. In the data preprocessing layer, symbolic data is converted into a format that can be interpreted by the computer (for example, a connection type identified as UDP is converted to a binary representation of 010. After this conversion, the data is normalized in order to keep the data uniform and manageable. In the CNN layer, this data is converted into an image, from which the CNN can extract features and produce a feature map. In the LTSM layer, this feature map is divided in to time steps time steps. New data that is seen as irrelevant or inconsistent can be filtered out (forgotten). 

To evaluate their model, the authors tested performance using the NSL-KDD dataset, which contains various sets of commonly used KDD data samples, and compared the results to other methods. The data set was divided into a training set and a testing set. Accuracy was used to measure the success of their model, defined as proportion of correctly classified samples to the total number of samples. The authors found that for the three samples they used, they reached a high accuracy on their training set (99.21 percent for the KDDTrain+  set) and high accuracy for their testing set (82.56 percent for the KDDTest+ set) while a third set (KDDTest-21) saw a more notable fluctuation in accuracy and a lower overall accuracy (67.55 percent). The authors compared accuracy scores of the KDDTest+ and KDDTest-21 and found that the accuracy of their model outperformed the others on both data sets. This includes a comparison to other models using a CNN, where they found the OCL model still outperformed on both data sets. Based on these results, the authors conclude that their model can be a powerful tool for intrusion detection. 

In (Nguyen), researchers proposed a similar strategy of using a CNN to detect a Denial of Service attack (DoS). Researchers identified two main difficulties with previous studies in this area, accuracy and execution time (especially when dealing with the large amount of data of network data). The researchers sought to address the issue of rising security concerns from rapidly expanding networks by creating an IDS based on a convolutional neural network, called IDS-CNN. The researchers note the previous popularity of machine learning methods, such as K-nearest neighbors, support vector machines, and Naïve Bayes classifiers, in detecting DoS attacks. 

The authors proposed a CNN-based IDS that is divided into four layers. The first layer, data collection, receives real-time network traffic data. In testing, this layer was fed data from a common KDD data set. The second layer, data preprocessing, prepares the raw data for the CNN model via normalization. The normalized data is then converted into a matrix to be fed to the next layer. In the next layer, the “CNN based DoS Detection Model” (cite), in which a trained CNN is used to classify the input from the previous layer. Data can be classified into either normal traffic, or four types of malicious traffic. The last layer, decision making, a decision is made to block or allow the traffic. If the model decides the traffic is malicious, it can either block or reroute the traffic. At this point, this data can also be added to a knowledge database. 

The researchers decided to implement the CNN with several configuration in order to find the best one for their purpose. The final configuration was two convolutional layers and three fully connected layers. The final layer computes a class score that can be 5 options, each pertaining to five attack categories (normal, or four types of attacks). 

To test their model, researchers first started by running an experiment in which they varied the number of convolutional layers of the CNN from one to three, choosing the configuration with the most accurate results. They then varied the number of training iterations from 1000 to 10000 to 15000, again choosing the configuration with the best results. For their input data, the researchers used 1 million data entries from the KDD dataset. From this, 70 percent are used for training and the remaining 30 percent are used for testing. The accuracy and execution time of this model is compared to three other models, SVM, KNN, and Naïve Bayes. 
When the models were compared, the researchers’ CNN model was found to be more accurate than the other machine learning techiques, reaching an accuract of 99.87 percent. This was much more accurate than the Naïve Bayes model (54.42 percent), and also more accurate than the SVM and KNN models (90.96 percent and 94.12 percent, respectively. Execution time was also significantly faster than both SVM and KNN models but was a bit slower than the Naïve Bayes model. 


\subsection{LTSM}

In Yang, researchers proposed a novel IDS based on Attention-Long Short Term Memory (LTSM) to improve upon the accuracy and precision of standard CNN, Recurrent Neural Networks (RNN), and LTSM algorithms. Similar to the previous paper, the authors build off of previous work building IDS with various machine learning techniques. The authors recognized that this previous work did not address some key problems (generalization, robustness, and false alarm rates), which they sought to address in this paper. 

To achieve this, the authors proposed an IDS based on Attention-LTSM. The attention-LTSM model is similar to a standard LTSM model but has the added ability to add weights to extracted data and thereby better identify critical data. It also provides the added benefit of removing redundant data and decreasing computations needed (Yang). Like models in previous papers, the model consists of multiple layers. Input and output layers are responsible for preprocessing data and outputting results. The preprocessed input data is handed to an LTSM layer, which calculates the attention probability of each feature, which is used to measure the importance of a given feature. This data is then handed off to a classification layer. The overall structure of their model is to normalize the data, hand that data to the Attention-LTSM module, then the classified data is given to a response module that responds to the intrusion based on the classification. 

To test their model, KDD-CUP99 data was used as the test data set, divided into training and testing data. The authors decided to include some attack types not seen in the training data to more accurately simulate a real-world scenario and evaluate the robustness of the model. 

The authors’ model was measured for accuracy and recall (measuring the percentage of false positives), and compared to three other models (CNN, RNN, LTSM). Results showed that the authors’ attention-LTSM model was more accurate than the three models, and also produced fewer false positives. 

